\titreexercice{1.a}{Marion L.}
\noindent On munit $\mathcal{M}_n(\mathbb{R})$ de son produit scalaire canonique.

\noindent Soit $A$ et $B$ deux matrices symétrique réelles telles que $A^2=B^2$.

\noindent On souhaite montrer dans cet exercice qu'il existe $P$ orthogonal tel que $PA = B$.

\noindent On admet que pour $D$ une matrice diagonale, $\mathrm{rg}(D) = \mathrm{rg}(D^2)$.

\vspace{5pt}
1. Soit $X \in \mathcal{M}_{n,1}(\mathbb{R})$. Etablir que $(AX)^\top AX = (BX)^\top BX$.

\vspace{5pt}
2. On suppose que $0$ n'est pas valeur propre de $A$.

\vspace{5pt}
\;\;\; 2.a Justifier que $A$ est inversible, et montrer que $BA^{-1}$ est orthogonale.

\vspace{5pt}
\;\;\; 2.b Conclure.

\vspace{5pt}
3. On suppose que $0$ est valeur propre de $A$ d'ordre de multiplicité $p$.

\vspace{5pt}
\;\;\; 3.a Montrer que $\mathrm{Im}(A) = \mathrm{Im}(A^2)$, puis que $\mathrm{Im}(A) = \mathrm{Im}(B)$.

\vspace{5pt}
\;\;\; 3.b Montrer que $\mathrm{Ker}(A) = \mathrm{Ker}(B)$, puis $0$ est valeur propre de $B$ d'un ordre de multiplicité que l'on déterminera.

\vspace{5pt}
\noindent Soit $R$ orthogonale telle que : $R^\top AR =
\begin{pmatrix}
0 & 0 \\
0 & \Delta \\
\end{pmatrix}
$
et $R^\top BR = 
\begin{pmatrix}
0 & 0 \\
0 & H \\
\end{pmatrix}
$ \\
où $(\Delta, H) \in S_{n-p}(\mathbb{R})^2$ sont inversibles et telles que $\Delta^2 = H^2$.

\vspace{5pt}
4. Montrer qu'il existe $P$ orthogonale telle que $PA = B$ (on s'aidera de 2.).

\vspace{5pt}
5. Démontrer la propriété précédemment admise.


\subetoiles



\titreexercice{1.b}{Marion L.}
\noindent Soit $\displaystyle \int_{1}^{+\infty} \frac {\mathrm{Arctan}(x)} {x^2}\mathrm{dx}$.

\vspace{5pt}
1. Etudier la convergence de cette intégrale.

\vspace{5pt}
2. Calculer cette intégrale.


\subetoiles
\columnbreak



\titreexercice{2.a}{Hugo D.}
\noindent Soit $E$ un sev de dimension $n \geqslant 2$ muni d'un produit scalaire et $S = \{x \in E \,/\; \|x\| = 1\}$.

\noindent Soit $u$ un endormophisme autoadjoint, et $q(x) = \langle x | u(x) \rangle$.

\vspace{5pt}
1. On se place dans le cas où n = 2. Soit $A =
\begin{pmatrix}
2 & 1 \\
1 & 2 \\
\end{pmatrix}
$, matrice de u.

Montrer que $u$ est autoadjoint, déterminer son spectre, et montrer que les vecteurs colonnes de $A$ forment une BON de $\mathbb{R}^2$.

\vspace{5pt}
2.a Montrer que $\displaystyle q(x) = \sum_{i = 1}^{n} x_i^2\lambda_i$,
avec $x_1, ..., x_n$ des vecteurs propres associés aux valeurs propres $\lambda_1, ..., \lambda_n$.

\vspace{5pt}
2.b En déduire que $\max\left\{q(x), x \in S\right\} = \lambda_n$.

\vspace{5pt}
3. Soit $E_k = Vect(e_1, ..., e_k)$.

\vspace{5pt}
\;\;\; 3.a Montrer que $\max \left\{q(x), x \in S \cap E_k \right\} = \lambda_k$.

\vspace{5pt}
\;\;\; 3.b Montrer que  $E_k = Vect(e_1, ..., e_k)$.

\vspace{5pt}
4. et 5. [non abordées]



\subetoiles



\titreexercice{2.b}{Hugo D.}
\noindent Soit $u_n$ une suite monotone.
On pose $\displaystyle M_n = \frac {1} {n} \sum_{k = 1}^{n} u_k$.

\vspace{5pt}
\noindent Montrer que $M_n$ est une suite monotone.



\subetoiles
\columnbreak



\titreexercice{3.a}{Cyrian D.}
\noindent Soit $x \in \mathbb{R}_+$. Soit $n \in \mathbb{N}^*$.
\noindent On pose
$\; f_n :
\begin{cases}
  \mathbb{R}_+  & \longrightarrow \mathbb{R}\\    
  x & \longmapsto \displaystyle \frac {1} {n + n^2x}
\end{cases}
$

\noindent On définit $\displaystyle f(x) = \sum_{n=1}^{+\infty}f_n(x)$ lorsque c'est possible.
\noindent On pose $g(x) = xf_n(x)$.

\vspace{5pt}
1. Montrer que $f$ est définie.

\vspace{5pt}
2.a Montrer que $f$ est monotone sur $\mathbb{R}_+^*$.

\vspace{5pt}
2.b Montrer que $f$ est $C^0$ sur $\mathbb{R}_+^*$.

\vspace{5pt}
3.a Montrer que $\displaystyle \lim_{x\to\infty} f(x) = 0$ est $C^0$ sur $\mathbb{R}_+^*$.

\vspace{5pt}
3.b Montrer que $\sum g_n$ CVN.

\vspace{5pt}
4. Montrer qu'il existe $A > 0$, tel que $\displaystyle f(x) \; \substack{\sim \\ x\to\infty} \; \frac {A} {x}$.

\vspace{5pt}
5. [non abordée]




\subetoiles




\titreexercice{3.b}{Cyrian D.}
\noindent Soit $A \in \mathcal{M}_n(\mathbb{R})$, $A =
\begin{pmatrix}
0 & 1 & 0 & . & 0 \\
1 & 0 & 1 & . & 0 \\
0 & 1 & 0 & . & 0 \\
. & . & . & . & 1 \\
0 & 0 & 0 & 1 & 0 \\
\end{pmatrix}
$.

\noindent Soit $\lambda \in \mathbb{R}$.

\vspace{5pt}
1. Montrer que $\mathrm{rg}(A-\lambda I_n)$ vaut $n-1$ ou $n$.

\vspace{5pt}
2. $A$ est-elle diagonalisable ?

\vspace{5pt}
3. Si $\lambda$ est valeur propre, quel est son ordre de multiplicité ?




\subetoiles
\columnbreak




\titreexercice{4.a}{Jules B.}
\noindent Soit $P \in \mathbb{R}_n[X]$.
\begin{itemize}
  \item On qualifie un polynôme $P$ de "positif" lorsque $\forall x \in \mathbb{R}$, $P(x) \geqslant 0$.
  \item De la même manière, un polynôme est dit "strictement positif" lorsque $\forall x \in \mathbb{R}$, $P(x) > 0$.
\end{itemize}

\vspace{5pt}
1. On pose $P(X) = X^2 - 2X + 1$, montrer que $P$ est positif.

On pose maintenant $Q = P'' + P' + P$, montrer que $Q$ est strictement positif.

\vspace{5pt}
2. Généralisation

\vspace{5pt}
\;\;\; 2.a Soit $P \in \mathbb{R}_n[X]$, on pose $\displaystyle Q = \sum_{k=0}^{2n}P^{(k)}$.

\;\;\; Exprimer $Q'$ en fonction de $Q$ et de $P$.

\vspace{10pt}
\;\;\; 2.b En posant $\displaystyle f : t \mapsto e^{-t}Q(t)$, montrer que $Q$ est strictement positif.

\vspace{10pt}
3. On pose maintenant $\displaystyle (\cdot | \cdot) : (P|Q) \mapsto \sum_{k=0}^{2n}(PQ)^{(k)}(0)$

\vspace{5pt}
\;\;\; 3.a Montrer que $(\cdot | \cdot)$ est un produit scalaire.

\vspace{5pt}
\;\;\; 3.b Déterminer une base orthogonale de $\mathbb{R}_n[X]$ pour ce produit scalaire.

\vspace{10pt}
4. On pose $u_n = d(X^n, \mathbb{R}_1[X])$. Déterminer $u_n$.

\vspace{10pt}
5. En supposant $\ln(n!) = n\ln(n) - n + o(n)$. Déterminer la nature de la série de terme général
$\displaystyle \left(\frac {1} {u_n^{3/2}}\right)_{n \geqslant 2}$.




\subetoiles




\titreexercice{4.b}{Jules B.}
\noindent Une urne contient $n$ boules numérotées de 1 à $n$.

\noindent On tire successivement k boules avec remise.

\noindent On note $X$ la variable aléatoire suivant le plus grand numéro ayant été tiré.

\vspace{10pt}
Déterminer la loi de $X$.

\vspace{10pt}
\noindent \textit{Indication} : Il faut utiliser et calculer $P(X \leqslant l)$ avec $l$ à déterminer.




\subetoiles
\columnbreak




\titreexercice{5.a}{Jean C.}
\noindent Soit $P \in \mathbb{R}_n[X]$. On pose $T_n(P) = (nX+1)P + (1-X^2)P'$.

\vspace{5pt}
1. Pour tout $k \in \llbracket 0;n \rrbracket$, calculer $T_n(X^k)$.

\vspace{5pt}
2. Montrer que $T_n$ est un endomorphisme.

\vspace{5pt}
3. On note $M_2$ la matrice de $T_2$ dans la base canonique. Déterminer $M_2$. $T_2$ est-elle diagonalisable ?

\vspace{5pt}
4. Soit $\lambda \in \mathbb{R}$ une valeur propre de $T_n$, et $\rho$ un vecteur propre associé à $\lambda$.

\vspace{5pt}
On pose $\forall x \in \lbrack -1; 1 \rbrack$,
$\displaystyle g_{\lambda}(x) = \int_{0}^{x} \frac {n t - \lambda + 1} {1 - t^2} \mathrm{d}t$.

\vspace{5pt}
On pose $h : x \mapsto \rho(x)e^{g_{\lambda}(x)}$. Montrer que $h$ est constante sur $\rbrack -1;1 \lbrack$.

\vspace{5pt}
5. Montrer que :
$\displaystyle \frac {nt - \lambda + 1} {1 - t^2} = \frac {-n - \lambda + 1} {2} \cdot \frac {1} {1 + t}
+ \frac {n - \lambda + 1} {2} \cdot \frac {1} {1 - t}$.

\vspace{5pt}
5. Calculer $g_\lambda$ sans intégrale.

\vspace{5pt}
6. On suppose $\lambda$ une valeur propre de $T_n$, et $\rho$ un vecteur propre associé.

Montrer que $\exists (k, \alpha, \beta) \in \mathbb{R}^3$ tel que $\rho = k(X-1)^\alpha(X+1)^\beta$.

\vspace{5pt}
7. Montrer que $T_n$ est diagonalisable.




\subetoiles




\titreexercice{5.b}{Jean C.}
\noindent Soit $f : 
\begin{cases}
  \lbrack 0; 1 \rbrack & \longrightarrow \mathbb{R} \\    
  (x,y) & \longmapsto x + x^2 - y + y^2
\end{cases}
$

\vspace{5pt}
\noindent On se place sur $\rbrack 0; 1 \lbrack$.

\vspace{5pt}
Déterminer les points critiques.

\vspace{5pt}
Montrer que f admet un maximum et le déterminer.

\vspace{5pt}
{\bfseries BONUS} : Extremum sur $\lbrack -1; 1 \rbrack$.