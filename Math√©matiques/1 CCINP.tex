\titreexercice{1}{ -- }
\noindent On pose, pour $(P,Q) \in \mathbb{R}_2[X]^2,$

\vspace{-15pt}
\begin{equation*}
    \langle {P, Q} \rangle = P(0)Q(0) + P(1)Q(1) + P(2)Q(2)
\end{equation*}
\vspace{-15pt}

1. Montrer que $\langle {P, Q} \rangle$ est un produit scalaire sur $\mathbb{R}_2[X]$.

\vspace{1pt}
2. Déterminer une base orthonormée sur $\mathbb{R}_2[X]$.



\subetoiles



\titreexercice{2}{ -- }
\noindent On considère l'espace euclidien $\mathbb{R}[X]$ ($n \geqslant 3$) muni du produit scalaire :

\vspace{-15pt}
\begin{equation*}
    \langle {P, Q} \rangle = \int_{-1}^{1}P(t)Q(t)dt
\end{equation*}
\begin{enumerate}
    \vspace*{-10pt}
    \item Déterminer $\displaystyle P_{\mathbb{R}_2[X]}(X^3)$.
    \vspace*{-5pt}
    \item Calculer $\displaystyle min\left\{\int_{-1}^{1}{(t^3 - (at^2 + bt + c))^2dt, (a,b,c) \in \mathbb{R}^3}\right\}$
\end{enumerate}



\subetoiles



\titreexercice{3}{ -- }
\noindent Soit $E = \mathcal{C}^1([0,1], \mathbb{R})$.

\noindent On pose pour tout $f \in E$ :
\begin{equation*}
    \|f\|_0 = \sqrt{(f(0))^2 + \int_{0}^{1} f'(t)dt}
\end{equation*}

1. Montrer que $||\cdot||_0$ est une norme.

2. Soit pour tout $n \in \mathbb{N}^*$, $\displaystyle f_n : x \in [0,1] \rightarrowtail \frac {\sin(n\pi x)} {n^2}$ \\
\indent \quad \,Montrer que $(f_n)_{n \in \mathbb{N}^*}$ converge pour la norme $\|\cdot\|_0$.