\titreexercice{1.a}{Ilian M.}
\noindent Soit $(a_n)_{n \in \mathbb{N}}$ définie par $a_0 = \frac {\pi} {2}$ et $a_{n+1} = \sin(a_n)$.

\vspace{5pt}
\noindent Nature de $\sum a_n^2$ ?



\subetoiles



\titreexercice{1.b}{Ilian M.}
\noindent Soit $(A, B, C) \in (M_2(\mathbb{R}))^3$.

\vspace{5pt}
\noindent On définit $\left[A, B\right] = AB - BA$ (ndlr, crochet de Lie).

\vspace{5pt}
\noindent Montrer que $\left[\left[A, B\right]^2, C\right] = 0$ avec deux méthodes différentes.



\subetoiles



\titreexercice{1.c}{Ilian M.}
\noindent Soit $A \in GL_n(\mathbb{R})$.

\vspace{5pt}
\noindent Montrer qu'il existe $P \in \mathbb{R}_n[X]$ tel que $A^{-1} = P(A)$.
