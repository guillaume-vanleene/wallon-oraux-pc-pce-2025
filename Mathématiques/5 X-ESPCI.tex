\titreexercice{1}{Ilian M.}
\noindent On pose $\displaystyle \gamma(x) = \frac {1} {\sqrt{2\pi}} e^{-\frac{x^2}{2}}$
et on donne $\displaystyle \int_{\mathbb{R}}^{} \gamma(x)\mathrm{dx} = 1$.

\vspace{5pt}
\noindent On pose aussi $(X_1, ..., X_n)$ des variables aléatoires mutuellement indépendantes et telles que
$\displaystyle \forall i \in \llbracket 1;n \rrbracket, \mathbb{P}(X_i = 1) = \mathbb{P}(X_i = -1) = \frac {1} {2}$.

\vspace{5pt}
\noindent On exprime alors $\displaystyle S_n = \frac {\sum_{k=1}^{n}\lambda_k} {\sqrt{n}}$.

\vspace{5pt}
\noindent Montrer alors que
$\displaystyle \forall Q \in \mathbb{R}[X], \mathbb{E}(Q(S_n)) = \int_{\mathbb{R}}^{}Q(x)\gamma(x)\mathrm{dx}$.

\vspace{10pt}
\noindent \textit{Indication} : s'intéresser aux monômes de degrés pairs et impairs.



\subetoiles



\titreexercice{2}{Pierre Q.}
\noindent Soit $P$ de degré $d$, $P \in \mathbb{Z}[X]$. On note $\lambda_1, ..., \lambda_d$ ses racines complexes.

\vspace{5pt}
\noindent On suppose $\forall k \in \llbracket 1;d\, \rrbracket, |\lambda_k| \leqslant 1$.
On note $\displaystyle f(n) = \sum_{k=1}^{d} \lambda_k^n \quad (n \in \mathbb{N})$.

\vspace{5pt}
1. Montrer que $f$ est à valeurs entières.

\vspace{5pt}
2. Montrer que $\exists p \in \mathbb{N}^*, \exists n_0 \in \mathbb{N} \ / \ \forall n \geqslant n_0, f(p+n) = f(p)$.

\vspace{5pt}
3. Montrer que les $\lambda_k$ sont soit nuls, soit racines de l'unité.