\titreexercice{1.a}{Raphaël F.}
\noindent Soit $M \in \mathcal{M}_n(\mathbb{R})$, telle que $M^n = O_n$.

\vspace{5pt}
1. Montrer que si $M$ est symétrique, alors $M = O_n$.

\vspace{5pt}
2. Montrer que si $MM^\top = M^\top M$, alors $M = O_n$.



\subetoiles



\titreexercice{1.b}{Raphaël F.}
\noindent On pose $\displaystyle H(x) = \int_{0}^{+\infty} \frac {\ln(t)} {x^2 + t^2}\mathrm{dt}$.

\vspace{5pt}
1. Donner le domaine de définition de $H$.

\vspace{5pt}
2. Calculer $H(1)$.

\vspace{5pt}
3. Trouver une expression de $H$ (ndlr, sans l'intégrale).




\subetoiles



\titreexercice{2.a}{Gaspard V.}
\noindent Soient $X$ et $Y$ deux variables aléatoires indépendantes telles que :
\begin{itemize}
  \item $X(\Omega) = Y(\Omega) = \mathbb{N}$.
  \item $\displaystyle \forall k \in \mathbb{N}, \; \mathbb{P}(X = k) = \mathbb{P}(Y = k) = \frac {1+a^k} {4k!}$
\end{itemize}


\vspace{5pt}
1. Déterminer $a$.

\vspace{5pt}
2. Déterminer l'espérance de $X$.

\vspace{5pt}
3. Déterminer la loi de $X + Y$.




\subetoiles
\columnbreak



\titreexercice{2.b}{Gaspard V.}
\noindent Soit $E$ un ev de dimension finie tel que $\dim(E) \geqslant 2$.
\noindent Soient $f$ et $g$ deux \\ endomorphismes de E vérifiant :
\begin{itemize}
  \item $f \circ f = g \circ g = Id_E$.
  \item $f \circ g + g \circ f = O_{\mathcal{L}(E)}$.
\end{itemize}

\vspace{5pt}
1. Montrer que $f$ et $g$ sont des automorphismes diagonalisables.

\vspace{5pt}
2. Montrer que les deux seules valeurs propres possibles pour $f$ et $g$ appartient à $\{-1; 1\}$.

\vspace{5pt}
3. Soit $u :
\begin{cases}
  \mathrm{Ker}(f - Id_E) & \longrightarrow \mathrm{Ker}(f + Id_E) \\    
  x & \longmapsto \displaystyle g(x)
\end{cases}
$

Montrer que $u$ est un isomorphisme et en déduire que la dimension de $E$ est paire.

\vspace{5pt}
4. [non abordée]



\subetoiles


\titreexercice{2.a}{Ilyes B.}
\noindent Soit $(X_n)$ une suite d'évènement indépendants suivant une loi de Bernouilli $p \in \lbrack 0;1 \rbrack$.

\vspace{5pt}
1. On pose $U_k = X_kX_{k+1}$. Déterminer la loi, l'espérance, et la variance des $Y_k$.

\vspace{5pt}
2. On pose $\displaystyle S_n = \sum_{k=1}^{N}Y_k$. Déterminer la loi, l'espérance, et la variance de $S_n$.

\vspace{5pt}
3. Montrer que $\mathbb{P}\left( |F_n - p| \geqslant 0 \right) \xrightarrow[k \rightarrow +\infty]{} 0$, où $\displaystyle F_n = \frac {\sum_{k=1}^{n}S_k} {n}$.



\subetoiles



\titreexercice{2.b}{Ilyes B.}
\noindent On pose $\displaystyle f : x \mapsto \sum_{n=1}^{+\infty} \frac {1} {n + n^2x}$.

\vspace{5pt}
1. Montrer que $f$ est bien définie sur $\rbrack 0; +\infty \lbrack$. Etudier sa continué sur $\rbrack 0; +\infty \lbrack$.

\vspace{5pt}
2. Montrer que $f$ est $C^1$.

\vspace{5pt}
3. Donner un équivalent de $f$ en 0.



\subetoiles
\columnbreak



\titreexercice{3.a}{Paul C.}
\noindent Soit $X \thicksim P(\lambda)$, $\lambda > 0$.

\vspace{5pt}
1. Déterminer l'espérance de $\exp(tX)$, avec $t > 0$. Préciser son existence.

\vspace{5pt}
2. Soit $n \in \mathbb{N}^*$. Montrer que $\forall t > 0$, $\mathbb{P}(X \geqslant n) \leqslant \mathrm{e}^{-tn}\,\mathbb{E}(\exp(tX))$.

\vspace{5pt}
3. Montrer que $\displaystyle \sum_{k = n}^{+\infty} \frac {\lambda_k} {k!} \leqslant \left( \frac {\lambda} {n} \right)^ne^n$ ($n \in \mathbb{N}^*$).



\subetoiles



\titreexercice{3.b}{Paul C.}
\noindent Soit $x \in \mathbb{R}^n$, $x = (x_1, ..., x_n)$, $(a,b) \in \mathbb{R}^2$.

\noindent On pose $\displaystyle f(x) = (a-b)^2 + \sum_{i=1}^{n}(x_i-a)^2 + \sum_{i=1}^{n}(x_i-b)^2 + \sum_{i \leqslant i,j \leqslant n}^{n}(x_i-x_j)^2$.

\noindent Pour tout $x \in \mathbb{R}$, on note $\displaystyle \|y\| = \sqrt{\langle y|y \rangle}$.

\vspace{10pt}
1. Montrer que $\forall x \in \mathbb{R}^n$, $\displaystyle f(x) \geqslant \|x-u\|^2 +  \|x-v\|^2 \geqslant (\|x\| - \|u\|)^2 + (\|x\| - \|v\|)^2$
où $
\begin{cases}
  u = (a, ..., a) & \\    
  v = (b, ..., b) &
\end{cases}$

\vspace{5pt}
2. Montrer que $\exists R > 0 \;/\; \forall x \in \mathbb{R}^n, \|x\| > R \Rightarrow f(x) > f(0)$.

\vspace{5pt}
3. Déterminer le minimum global de $f$ sur $\mathbb{R}^n$.



\subetoiles



\titreexercice{4.a}{Tristan D.}
\noindent Déterminer les fonctions développables en série entière vérifiant l'équation différentielle
\begin{equation*}
  x^2(1-x)y''(x) - x(1+x)y'(x) + y(x) = 0
\end{equation*}



\subetoiles
\columnbreak



\titreexercice{4.b}{Tristan D.}
\noindent Soit $X$ et $Y$ deux variables aléatoires à valeurs dans $\mathbb{N}$.

\noindent On donne la loi du couple $(X,Y)$ :
$\displaystyle \forall (i,j) \in \mathbb{N}^2, \mathbb{P}((X,Y) = (i,j)) = \frac {1} {e2^{i+1}j!}$.

\vspace{5pt}
1. Déterminer les lois de $X$ et $Y$.

\vspace{5pt}
2. Montrer que $X+1$ suit une loi géométrique et en déduire l'espérance et la variance de $X$.

\vspace{5pt}
3. $X$ et $Y$ sont-elles indépendantes ?

\vspace{5pt}
4. Calculer $\mathbb{P}(X=Y)$.



\subetoiles



\titreexercice{5.a}{Lucy D.}
\noindent Soit $A \in \mathcal{M}_n(\mathbb{R})$ une matrice non nulle et
$\rho(M) = \mathrm{Tr}(A)M - \mathrm{Tr}(M)A$.

\vspace{5pt}
1. Déterminer $\mathrm{Ker}(\rho)$ et le rang de $\rho$.

\vspace{5pt}
2. $\rho$ est-elle diagonalisable ?



\subetoiles



\titreexercice{5.b}{Lucy D.}
Soit $\alpha \in \mathbb{R}$, $(f_n)_{n \geqslant 1}$ une suite de fontions définies par $f_n : 
\begin{cases}
  \mathbb{R}^+ & \longrightarrow \mathbb{R} \\    
  x & \longmapsto n^\alpha x e^{-nx}
\end{cases}
$.

\vspace{5pt}
1. Etudier la convergence uniforme en fonction de $\alpha \in \mathbb{R}$.

\vspace{5pt}
2. Calculer $\displaystyle \lim_{n\to\infty} \int_{0}^{+\infty}n^\alpha x e^{-nx}\mathrm{d}x$.



\subetoiles



\titreexercice{6.a}{Marion L.}
\noindent Soit $X$ et $Y$ deux variables aléatoires indépendantes et indentiquement distribuées, qui suivent
une loi uniforme sur $\{0, 1\}$.

\noindent Soit $A(X, Y) = 
\begin{pmatrix}
X & 1 \\
1 & Y \\
\end{pmatrix}
$.

\vspace{5pt}
1. Déterminer la probabilité que le polynôme caractéristique de $A(X,Y)$ soit scindé à racines simples.

\vspace{5pt}
2. Déterminer la probabilité que $A(X, Y)$ soit diagonalisable.

\vspace{5pt}
3. Déterminer la probabilité que $A(X, Y)$ soit symétrique et ?.



\subetoiles
\columnbreak



\titreexercice{6.b}{Marion L.}
\noindent Soit $f : \mathbb{R} \mapsto \mathbb{R}$ deux fois dérivables telle que
$\forall x \in \mathbb{R}, f''(x) + f(-x) = x$.

\vspace{5pt}
\noindent Soit $g$ telle que $\forall x \in \mathbb{R}, g(x) = f(x) + f(-x)$.

\vspace{5pt}
1. Démontrer que $g$ est une solution paire de l'équation différentielle $y'' + y = 0$.

\vspace{5pt}
2. Déterminer les fonctions $f : \mathbb{R} \mapsto \mathbb{R}$ solutions de l'équation :
\begin{equation*}
  \forall x \in \mathbb{R}, f''(x) + f(-x) = x
\end{equation*}



\subetoiles



\titreexercice{7.a}{Matthieu R.}
\noindent Soit $E$ un ev de dimension $n$ et $B = (e_1, ..., e_n)$ une base othonormale $E$.

\noindent Soit $f$ un endomorphisme de $E$ tel que
\begin{equation*}
  \forall (i,j) \in \llbracket 1; n \rrbracket, i \ne j, \langle e_i | e_j \rangle = 0 \Rightarrow \langle f(e_i) | f(e_j) \rangle = 0
\end{equation*}

\vspace{5pt}
1. Montrer que $\forall i \in \llbracket 1; n \rrbracket, f(e_1) + f(e_2)$ et $f(e_1) - f(e_i)$ sont orthogonaux.

\vspace{5pt}
2. En déduire qu'il existe un $k \in \mathbb{R}$ tel que $\forall x \in E, \| f(x) \| = k \|x\|$.



\subetoiles



\titreexercice{7.b}{Matthieu R.}
\noindent On considère la série entière 
$\displaystyle \sum_{n=1}^{+\infty} \frac {(-1)^n} {(2n+1)(2n-1)} x^{2n+1}$.

\vspace{5pt}
\noindent Trouver la rayon de convergence $R$, puis pour $|x| < R$, calculer la somme de cette série.
