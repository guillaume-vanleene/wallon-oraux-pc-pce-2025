\titreexercice{1.a}{Raphaël F.}
\noindent Soit $M \in \mathcal{M}_n(\mathbb{R})$, telle que $M^n = O_n$.

\vspace{5pt}
1. Montrer que si $M$ est symétrique, alors $M = O_n$.

\vspace{5pt}
2. Montrer que si $MM^\top = M^\top M$, alors $M = O_n$.



\subetoiles



\titreexercice{1.b}{Raphaël F.}
\noindent On pose $\displaystyle H(x) = \int_{0}^{+\infty} \frac {\ln(t)} {x^2 + t^2}\mathrm{dt}$.

\vspace{5pt}
1. Donner le domaine de définition de $H$.

\vspace{5pt}
2. Calculer $H(1)$.

\vspace{5pt}
3. Trouver une expression de $H$ (ndlr, sans l'intégrale).




\subetoiles



\titreexercice{2.a}{Gaspard V.}
\noindent Soient $X$ et $Y$ deux variables aléatoires indépendantes telles que :
\begin{itemize}
  \item $X(\Omega) = Y(\Omega) = \mathbb{N}$.
  \item $\displaystyle \forall k \in \mathbb{N}, \; \mathbb{P}(X = k) = \mathbb{P}(Y = k) = \frac {1+a^k} {4k!}$
\end{itemize}


\vspace{5pt}
1. Déterminer $a$.

\vspace{5pt}
2. Déterminer l'espérance de $X$.

\vspace{5pt}
3. Déterminer la loi de $X + Y$.




\subetoiles
\columnbreak



\titreexercice{2.b}{Gaspard V.}
\noindent Soit $E$ un ev de dimension finie tel que $\dim(E) \geqslant 2$.
\noindent Soient $f$ et $g$ deux \\ endomorphismes de E vérifiant :
\begin{itemize}
  \item $f \circ f = g \circ g = Id_E$.
  \item $f \circ g + g \circ f = O_{\mathcal{L}(E)}$.
\end{itemize}

\vspace{5pt}
1. Montrer que $f$ et $g$ sont des automorphismes diagonalisables.

\vspace{5pt}
2. Montrer que les deux seules valeurs propres possibles pour $f$ et $g$ appartient à $\{-1; 1\}$.

\vspace{5pt}
3. Soit $u :
\begin{cases}
  \mathrm{Ker}(f - Id_E) & \longrightarrow \mathrm{Ker}(f + Id_E) \\    
  x & \longmapsto \displaystyle g(x)
\end{cases}
$

Montrer que $u$ est un isomorphisme et en déduire que la dimension de $E$ est paire.

\vspace{5pt}
4. [non abordée]



\subetoiles


\titreexercice{2.a}{Ilyes B.}
\noindent Soit $(X_n)$ une suite d'évènement indépendants suivant une loi de Bernouilli $p \in \lbrack 0;1 \rbrack$.

\vspace{5pt}
1. On pose $U_k = X_kX_{k+1}$. Déterminer la loi, l'espérance, et la variance des $Y_k$.

\vspace{5pt}
2. On pose $\displaystyle S_n = \sum_{k=1}^{N}Y_k$. Déterminer la loi, l'espérance, et la variance de $S_n$.

\vspace{5pt}
3. Montrer que $\mathbb{P}\left( |F_n - p| \geqslant 0 \right) \xrightarrow[k \rightarrow +\infty]{} 0$, où $\displaystyle F_n = \frac {\sum_{k=1}^{n}S_k} {n}$.



\subetoiles



\titreexercice{2.b}{Ilyes B.}
\noindent On pose $\displaystyle f : x \mapsto \sum_{n=1}^{+\infty} \frac {1} {n + n^2x}$.

\vspace{5pt}
1. Montrer que $f$ est bien définie sur $\rbrack 0; +\infty \lbrack$. Etudier sa continué sur $\rbrack 0; +\infty \lbrack$.

\vspace{5pt}
2. Montrer que $f$ est $C^1$.

\vspace{5pt}
3. Donner un équivalent de $f$ en 0.



\subetoiles
\columnbreak



\titreexercice{3.a}{Paul C.}
\noindent Soit $X \thicksim P(\lambda)$, $\lambda > 0$.

\vspace{5pt}
1. Déterminer l'espérance de $\exp(tX)$, avec $t > 0$. Préciser son existence.

\vspace{5pt}
2. Soit $n \in \mathbb{N}^*$. Montrer que $\forall t > 0$, $\mathbb{P}(X \geqslant n) \leqslant \mathrm{e}^{-tn}\,\mathbb{E}(\exp(tX))$.

\vspace{5pt}
3. Montrer que $\displaystyle \sum_{k = n}^{+\infty} \frac {\lambda_k} {k!} \leqslant \left( \frac {\lambda} {n} \right)^ne^n$ ($n \in \mathbb{N}^*$).



\subetoiles



\titreexercice{3.b}{Paul C.}
\noindent Soit $x \in \mathbb{R}^n$, $x = (x_1, ..., x_n)$, $(a,b) \in \mathbb{R}^2$.

\noindent On pose $\displaystyle f(x) = (a-b)^2 + \sum_{i=1}^{n}(x_i-a)^2 + \sum_{i=1}^{n}(x_i-b)^2 + \sum_{i \leqslant i,j \leqslant n}^{n}(x_i-x_j)^2$.

\noindent Pour tout $x \in \mathbb{R}$, on note $\displaystyle \|y\| = \sqrt{\langle y|y \rangle}$.

\vspace{10pt}
1. Montrer que $\forall x \in \mathbb{R}^n$, $\displaystyle f(x) \geqslant \|x-u\|^2 +  \|x-v\|^2 \geqslant (\|x\| - \|u\|)^2 + (\|x\| - \|v\|)^2$
où $
\begin{cases}
  u = (a, ..., a) & \\    
  v = (b, ..., b) &
\end{cases}$

\vspace{5pt}
2. Montrer que $\exists R > 0 \;/\; \forall x \in \mathbb{R}^n, \|x\| > R \Rightarrow f(x) > f(0)$.

\vspace{5pt}
3. Déterminer le minimum global de $f$ sur $\mathbb{R}^n$.