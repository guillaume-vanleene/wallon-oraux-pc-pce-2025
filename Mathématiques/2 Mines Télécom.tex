\titreexercice{1.a}{Raphaël F.}
\noindent Soit $M \in \mathcal{M}_n(\mathbb{R})$, telle que $M^n = O_n$.

\vspace{5pt}
1. Montrer que si $M$ est symétrique, alors $M = O_n$.

\vspace{5pt}
1. Montrer que si $MM^\top = M^\top M$, alors $M = O_n$.



\subetoiles



\titreexercice{1.b}{Raphaël F.}
\noindent On pose $\displaystyle H(x) = \int_{0}^{+\infty} \frac {ln(t)} {x^2 + t^2} dt$.

\vspace{5pt}
1. Donner le domaine de définition de $H$.

\vspace{5pt}
2. Calculer $H(1)$.

\vspace{5pt}
3. Trouver une expression de $H$ (ndlr, sans l'intégrale).




\subetoiles



\titreexercice{2.a}{Gaspard V.}
\noindent Soient $X$ et $Y$ deux variables aléatoires indépendantes telles que :
\begin{itemize}
  \item $X(\Omega) = Y(\Omega) = \mathbb{N}$.
  \item $\displaystyle \forall k \in \mathbb{N}, \; \mathbb{P}(X = k) = \mathbb{P}(Y = k) = \frac {1+a^k} {4k!}$
\end{itemize}


\vspace{5pt}
1. Déterminer $a$.

\vspace{5pt}
2. Déterminer l'espérance de $X$.

\vspace{5pt}
3. Déterminer la loi de $X + Y$.




\subetoiles
\columnbreak



\titreexercice{2.b}{Gaspard V.}
\noindent Soient $E$ un ev de dimension finie tel que $\dim(E) \geqslant 2$.
\noindent Soient $f$ et $g$ deux \\ endomorphismes de E vérifiant :
\begin{itemize}
  \item $f \circ f = g \circ g = Id_E$.
  \item $f \circ g + g \circ f = O_{\mathcal{L}(E)}$.
\end{itemize}

\vspace{5pt}
1. Montrer que $f$ et $g$ sont des automorphismes diagonalisables.

\vspace{5pt}
2. Montrer que les deux seules valeurs propres possibles pour $f$ et $g$ appartient à $\{-1; 1\}$.

\vspace{5pt}
3. Soit $u :  
\begin{cases}
  \mathrm{Ker}(f - Id_E) & \longrightarrow \mathrm{Ker}(f + Id_E) \\    
  x & \longmapsto \displaystyle g(x)
\end{cases}
$

Montrer que $u$ est un isomorphisme et en déduire que la dimension de $E$ est paire.

\vspace{5pt}
4. [non abordé]



\subetoiles