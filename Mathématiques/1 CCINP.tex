\titreexercice{1.a}{Marion L.}
\noindent On munit $\mathcal{M}_n(\mathbb{R})$ de son produit scalaire canonique.

\noindent Soit $A$ et $B$ deux matrices symétrique réelles telles que $A^2=B^2$.

\noindent On souhaite montrer dans cet exercice qu'il existe $P$ orthogonal tel que $PA = B$.

\noindent On admet que pour $D$ une matrice diagonale, $\mathrm{rg}(D) = \mathrm{rg}(D^2)$.

\vspace{5pt}
1. Soit $X \in \mathcal{M}_{n,1}(\mathbb{R})$. Etablir que $(AX)^\top AX = (BX)^\top BX$.

\vspace{5pt}
2. On suppose que $0$ n'est pas valeur propre de $A$.

\vspace{5pt}
\;\;\; 2.a Justifier que $A$ est inversible, et montrer que $BA^{-1}$ est orthogonale.

\vspace{5pt}
\;\;\; 2.b Conclure.

\vspace{5pt}
3. On suppose que $0$ est valeur propre de $A$ d'ordre de multiplicité $p$.

\vspace{5pt}
\;\;\; 3.a Montrer que $\mathrm{Im}(A) = \mathrm{Im}(A^2)$, puis que $\mathrm{Im}(A) = \mathrm{Im}(B)$.

\vspace{5pt}
\;\;\; 3.b Montrer que $\mathrm{Ker}(A) = \mathrm{Ker}(B)$, puis $0$ est valeur propre de $B$ d'un ordre de multiplicité que l'on déterminera.

\vspace{5pt}
\noindent Soit $R$ orthogonale telle que : $R^\top AR =
\begin{pmatrix}
0 & 0 \\
0 & \Delta \\
\end{pmatrix}
$
et $R^\top BR = 
\begin{pmatrix}
0 & 0 \\
0 & H \\
\end{pmatrix}
$ \\
où $(\Delta, H) \in S_{n-p}(\mathbb{R})^2$ sont inversibles et telles que $\Delta^2 = H^2$.

\vspace{5pt}
4. Montrer qu'il existe $P$ orthogonale telle que $PA = B$ (on s'aidera de 2.).

\vspace{5pt}
5. Démontrer la propriété précédemment admise.


\subetoiles



\titreexercice{1.b}{Marion L.}
\noindent Soit $\displaystyle \int_{1}^{+\infty} \frac {\mathrm{Arctan}(x)} {x^2}\mathrm{dx}$.

\vspace{5pt}
1. Etudier la convergence de cette intégrale.

\vspace{5pt}
2. Calculer cette intégrale.


\subetoiles
\columnbreak



\titreexercice{2.a}{Hugo D.}
\noindent Soit $E$ un sev de dimension $n \geqslant 2$ muni d'un produit scalaire et $S = \{x \in E \,/\; \|x\| = 1\}$.

\noindent Soit $u$ un endormophisme autoadjoint, et $q(x) = \langle x | u(x) \rangle$.

\vspace{5pt}
1. On se place dans le cas où n = 2. Soit $A =
\begin{pmatrix}
2 & 1 \\
1 & 2 \\
\end{pmatrix}
$, matrice de u.

Montrer que $u$ est autoadjoint, déterminer son spectre, et montrer que les vecteurs colonnes de $A$ forment une BON de $\mathbb{R}^2$.

\vspace{5pt}
2.a Montrer que $\displaystyle q(x) = \sum_{i = 1}^{n} x_i^2\lambda_i$,
avec $x_1, ..., x_n$ des vecteurs propres associés aux valeurs propres $\lambda_1, ..., \lambda_n$.

\vspace{5pt}
2.b En déduire que $\max\left\{q(x), x \in S\right\} = \lambda_n$.

\vspace{5pt}
3. Soit $E_k = Vect(e_1, ..., e_k)$.

\vspace{5pt}
\;\;\; 3.a Montrer que $\max \left\{q(x), x \in S \cap E_k \right\} = \lambda_k$.

\vspace{5pt}
\;\;\; 3.b Montrer que  $E_k = Vect(e_1, ..., e_k)$.

\vspace{5pt}
4. et 5. [non abordées]



\subetoiles



\titreexercice{2.b}{Hugo D.}
\noindent Soit $u_n$ une suite monotone.
On pose $\displaystyle M_n = \frac {1} {n} \sum_{k = 1}^{n} u_k$.

\vspace{5pt}
\noindent Montrer que $M_n$ est une suite monotone.



\subetoiles
\columnbreak



\titreexercice{3.a}{Cyrian D.}
\noindent Soit $x \in \mathbb{R}_+$. Soit $n \in \mathbb{N}^*$.
\noindent On pose
$\; f_n :
\begin{cases}
  \mathbb{R}_+  & \longrightarrow \mathbb{R}\\    
  x & \longmapsto \displaystyle \frac {1} {n + n^2x}
\end{cases}
$

\noindent On définit $\displaystyle f(x) = \sum_{n=1}^{+\infty}f_n(x)$ lorsque c'est possible.
\noindent On pose $g(x) = xf_n(x)$.

\vspace{5pt}
1. Montrer que $f$ est définie.

\vspace{5pt}
2.a Montrer que $f$ est monotone sur $\mathbb{R}_+^*$.

\vspace{5pt}
2.b Montrer que $f$ est $C^0$ sur $\mathbb{R}_+^*$.

\vspace{5pt}
3.a Montrer que $\displaystyle \lim_{x\to\infty} f(x) = 0$ est $C^0$ sur $\mathbb{R}_+^*$.

\vspace{5pt}
3.b Montrer que $\sum g_n$ CVN.

\vspace{5pt}
4. Montrer qu'il existe $A > 0$, tel que $\displaystyle f(x) \; \substack{\sim \\ x\to\infty} \; \frac {A} {x}$.

\vspace{5pt}
5. [non abordée]




\subetoiles




\titreexercice{3.b}{Cyrian D.}
\noindent Soit $A \in \mathcal{M}_n(\mathbb{R})$, $A =
\begin{pmatrix}
0 & 1 & 0 & . & 0 \\
1 & 0 & 1 & . & 0 \\
0 & 1 & 0 & . & 0 \\
. & . & . & . & 1 \\
0 & 0 & 0 & 1 & 0 \\
\end{pmatrix}
$.

\noindent Soit $\lambda \in \mathbb{R}$.

\vspace{5pt}
1. Montrer que $\mathrm{rg}(A-\lambda I_n)$ vaut $n-1$ ou $n$.

\vspace{5pt}
2. $A$ est-elle diagonalisable ?

\vspace{5pt}
3. Si $\lambda$ est valeur propre, quel est son ordre de multiplicité ?