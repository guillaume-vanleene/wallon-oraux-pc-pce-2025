\titreexercice{1.a}{Armel D.}
\noindent Soit $a \in \mathbb{R}^*$. On pose $\displaystyle M = 
\begin{pmatrix}
0                 & a               & a^2\\
\nicefrac{1}{a}   & 0               & a  \\
\nicefrac{1}{a^2} & \nicefrac{1}{a} & 0  \\
\end{pmatrix}
$.

\vspace{5pt}
1. Calculer $M^2$. En déduire que $M$ est inversible, et déterminer $M^{-1}$.

\vspace{5pt}
2. Sans utiliser le polynôme caractéristique, montrer que $M$ est diagonalisable.

Déterminer ses valeurs propres et leur ordre de multiplicité.

\vspace{5pt}
3. Calculer $M^n$.

\textit{Indication} : on utilisera le théorème de la division euclidienne.

\vspace{5pt}
4. Soit $N \in \mathbb{N}^*$. On pose $\displaystyle S_N = \sum_{n=0}^{N} \frac{M^n}{n!}$.

Montrer que $(S_N)_{N \in \mathbb{N}}$ converge vers une limite $L$ finie et déterminer cette limite.



\subetoiles



\titreexercice{1.b}{Armel D.}
\noindent Soit $x \in \mathbb{R}$.
On pose $\displaystyle F(x) = \int_{-\infty}^{+\infty} \frac {dt} {(1+t^2)(1+ixt)}$.

\vspace{5pt}
1. Montrer que $F$ est bien définie et continue.

\vspace{5pt}
2. Pour $x \in \mathbb{R}$, montrer que $F(x) \in \mathbb{R}$.

\vspace{5pt}
3. Déterminer une expression de $F$ sans symbole d'intégrale.



\subetoiles
\columnbreak



\titreexercice{2.a}{Guillaume P.}
\noindent Soit $\alpha\in \mathbb{R}$.

\vspace{5pt}
1. Déterminer le développement asymptotique à deux termes de précision \\ près de
$\displaystyle \left( 1 + \frac {\alpha} {n}\right)^n$.

\vspace{5pt}
\noindent Soit $n \in \mathbb{N}$ \\
\noindent On considère $E_n = \{1, ..., n\} = \llbracket 1;n \rrbracket$. \\
\noindent On considère également $\Omega_n = {E_n}^{E_n}$
(l'ensemble des applications de $E_n$ dans $E_n$) muni de la loi uniforme. \\
\noindent On introduit $\forall k \in E_n$, la va $X_{k,n} : \Omega_n \rightarrow \{0,1\}$,
indicatrice de l'évènement $\{g \in \Omega_n; k \in g(E_n)\}$. \\
\noindent On introduit $Y_n : \Omega_n \rightarrow \mathbb{N}$ la va qui à tout $g$ de $\Omega_n$
associe $|g(E_n)|$ (cardinal de $g(E_n)$).

\vspace{5pt}
2. Déterminer la loi de $X_{k,n}$.

\vspace{5pt}
3. Déterminer $\mathbb{E}(Y_n)$.

\vspace{5pt}
4. Soit $(k,l) \in E_n^2$. Déterminer la loi du couple $(X_{k,n}, X{l,n})$.

En déduire :
$\textrm{COV}(X_{k,n}, X_{l,n}) = \left( 1 - \frac {2} {n}\right) ^n + ... \left(1 + \frac {...} {n}\right)^{2n}$.

\vspace{5pt}
5. Déterminer $\mathbb{V}(Y_n)$.

\vspace{5pt}
6. Déterminer un équivalent de $\mathbb{V}(Y_n)$.



\subetoiles



\titreexercice{2.b}{Guillaume P.}
\noindent Soit $f \in C^{\infty}(\mathbb{R}, \mathbb{R})$.

\noindent On suppose que $\sum \frac {f^{(n)}} {n!}$ CVU sur tout segment de $\mathbb{R}$.

\noindent Déterminer une expression de la somme de cette série.



\subetoiles
\columnbreak