\titreexercice{3}{C D.}
Démontrer l'équation de Navier-Stokes :

\vspace{-15pt}
\begin{equation*}
    \mu \frac {D \boldsymbol {v}} {Dt}
    = \mu \left(\frac {\partial \boldsymbol {v}} {\partial t} + (\boldsymbol {v} \cdot \boldsymbol {grad}) \boldsymbol {v} \right)
    = - \boldsymbol {grad} \, P + \mu \boldsymbol {g} + \eta \Delta \boldsymbol {v}
\end{equation*}

L'équation de Schrödinger :
\begin{equation*}
    ih \frac{\partial \boldsymbol {\psi}} {\partial t}
    = \hat {H} \boldsymbol {\psi}
    = - \frac{\hbar^2}{2m} \Delta \boldsymbol {\psi} + V \boldsymbol {\psi}
\end{equation*}

Ainsi que les équations de Maxwell :
\begin{equation*}
    div \boldsymbol {E} = \frac {\rho} {\epsilon_0}
\end{equation*}
\begin{equation*}
    div \boldsymbol {B} = 0
\end{equation*}
\begin{equation*}
    \boldsymbol {rot} \boldsymbol {E} = - \frac {\partial \boldsymbol {B}} {\partial t}
\end{equation*}
\begin{equation*}
    \boldsymbol {rot} \boldsymbol {B} = \mu_0 \boldsymbol {j} + \mu_0 \epsilon_0 \frac {\partial \boldsymbol {E}} {\partial t}
\end{equation*}
